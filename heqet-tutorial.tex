\documentclass{article}
\usepackage{booktabs}
\usepackage{fancyhdr}
\pagestyle{empty}
\usepackage{graphicx}
 \usepackage{color}
 \usepackage{transparent}
 \usepackage{hyperref}
 \usepackage{lastpage}

\pagestyle{fancy}
\lhead{} % Top left header
\chead{Heqet Tutorial} % Top center head
\rhead{} % Top right header
\lfoot{} % Bottom left footer
\cfoot{} % Bottom center footer
\rfoot{ \thepage/\protect\pageref{LastPage}} % Bottom right footer \protect\pageref{LastPage}

\newcommand{\prob}[1]{\noindent\textbf{#1.}}

\begin{document}
\pagestyle{empty}
\begin{center}
\vspace{3cm}
\scalebox{1.5}{{\Huge Heqet Tutorial}}

\vspace{2cm}
{\Large By Isaac Reilly}

\vspace{0.15cm}
{\large advised by Donya Quick}

\vspace{0.1cm}
{senior project for the Yale University major in Computer Science}

\vspace{13cm}

{\small fall 2015}

\end{center}

\pagebreak

\tableofcontents
\newpage

\begin{section}{Heqet for Euterpea Users}
Given a Euterpea music expression, creating a score is
 straightforward. You will need to import \verb+Heqet+ 
 and \verb+Heqet.Input.Euterpea+ and then apply the following functions:
First, to convert it to a Heqet music value, 
you should use \verb+fromEu+ or \verb+fromEu1+, which take a Euterpea \verb+Music Pitch+ or a \verb+Music Note1+ respectively. Now you can print a score to
 stdout with the functions \verb+quickScore+ or \verb+quickLine+. \verb+quickScore+ produces a piano rendition, while \verb+quickLine+ writes a single staff. 
 
 The reason that these functions perform IO is that the 
 simplest way to use Heqet is to write a Haskell program which outputs your Lilypond code and 
 pipe it directly into Lilypond, which you can do by running the included \verb+heqet.sh+ bash script with your Haskell program filename as an argument. Alternatively, just load your program into GHCi and paste the output code into a Lilypond file.

Now, let's try something more complicated. Suppose you have a score that has multiple instruments and you want to write them on separate staves with appropriate labels. If your Euterpea music value has instrument modifiers, you can just use \verb+writeScore+ for your output function, since Euterpea instruments are converted automatically. If you don't have instrument modifiers, or if your instruments are beyond the basic set of instruments currently recognized, you'll have to assign instruments in Heqet. The tools to manipulate music are mostly lenses, best used with the \verb+Control.Lens+ package. Lenses are a way to modify portions of a larger piece of data, for example, changing the instrument assigned to some of the notes of a music value. 

By the way, this is a good time to mention that in Heqet most properties of music are recorded on every single note they impact. So, every note has a field for its \verb+Maybe Instrument+. This way, you can rearrange the music as much as you want without worrying about handling the transitions between states over the course of the music---if the instrument changes, then Heqet is supposed to automatically write this direction into the part.(Actually, this isn't implemented yet. An example that does work is handling slurs: a slurred is considered to be an articulation on all but the last note in the slur. If you cut a slurred phrase in half, both halves will stay slurred correctly.) 

Suppose we have a Heqet music value \verb+opus+.
\end{section}

\begin{section}{Heqet for Lilypond Users}

\begin{subsection}{Note input}

\end{subsection}

\begin{subsection}{Differences from Lilypond input}
There are several differences between the Heqet note-input domain-specific language and Lilypond input, for a variety of reasons. 
\begin{itemize}
\item You can enter notes with any rational duration with the \verb+\d+ syntax, for example \verb+c\d 4/5+ to make a note with a duration of $4/5$ of a WHAT. You can omit the denominator if it's $1$. This is currently the only way to enter notes of the durations needed for a tuplet. 

\item \verb+hz+

\item \verb+c-(+ for slurs (sorry)

\item no clefs or time signatures

\item functions, commands, \verb+\with+ . Lilypond parsing problems

\item only absolute entry at the moment. 

\item percussion notation \verb+\phh+ for hi-hat. When you enter notes, you don't need to specify that they must be rendered in a DrumStaff. They should be automatically rendered well, although support for percussion is currently minimal.

\item no way to manually write beams, as this is not something the musician should need to worry about.
\end{itemize}
\end{subsection}

\begin{subsection}{Making a score}

\end{subsection}

\begin{subsection}{Lilypond tweaks}

\end{subsection}

\end{section}

\begin{section}{Advanced topics}

\end{section}

\end{document}